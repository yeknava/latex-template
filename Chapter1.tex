\chapter{کلیات}
\section{مقدمه فصل}
%\section[title] {title\footnote{title ftnt}}
«پردازش ابری» یا «پردازش انبوه» یک فناوری جدید است که به‌تازگی از طرف شرکت‌های مختلف از جمله: مایکروسافت، گوگل، آمازون و آی‌بی‌ام عرضه شده است. در این سیستم به جای نصب چندین نرم‌افزار بر روی چند رایانه، تنها یک نرم‌افزار، یک بار اجرا و بارگذاری می‌شود و تمامی افراد از طریق یک خدمات بر خط\LTRfootnote{Online services} 
به آن دسترسی پیدا می‌کنند. به این پروسه «پردازش ابری»\LTRfootnote{Cloud Computing} 
می‌گویند.
\section{بیان مساله}
امروزه با پیشرفت روز افزون تلفن‌های هوشمند و تبلت‌ها به تعداد کاربران آنها نیز اضافه خواهد شد، تا جایی که تنها تا چند سال دیگر بیش از نیمی از انسان‌ها یک تلفن هوشمند خواهند داشت و همچنین با پیشرفت تکنولوژی‌ها در زمینه پردازش همه جانبه\LTRfootnote{Ubiquitous computing} 
به ازای هر نفر چندین دستگاه کامپیوتری وجود خواهد داشت.

\section{روش انجام تحقیق}
روش معادل فارسی کلمه متد\LTRfootnote{Method} 
و به معنی در پیش گرفتن راهی و یا معین کردن گام‌هایی است که برای رسیدن به هدفی می‌باید با نظمی خاص برداشت. طبیعت این گام‌ها و اوصاف تفصیلی آنها بستگی به هدف مطلوب و نحوه رسیدن به آن دارد.
پژوهش یا تحقیق عبارت است از یک سری گام‌های مدون که برای جمع آوری داده‌های مناسب  و تحلیل آنها به کار می‌رود تا در سایه‌ی این فرآیند مدون بر دانش محقق در زمینه خاص مورد پژوهش افزوده گردد. نوع پژوهش نوع تحقیق توصیفی بوده و هدف از آن بدست آوردن اطلاعات کافی در رابطه با مسئله بوده است.
\section{قلمرو تحقیق}
الف) دوره زمانی انجام تحقیق:
سال ۹۱-۹۲ شمسی
ب)مکان تحقیق:
تهران.
