\usepackage[top=35mm, bottom=30mm, left=30mm, right=35mm]{geometry} %تنظیم حاشیه‌های صفحه
\linespread{2.5} %تعداد ۲۲ خط در صفحه

\usepackage{hyperref} %ایجاد لینک در متن
\hypersetup{linkbordercolor={1 1 1}, urlbordercolor={1 1 1}} %سفید کردن مرز لینک‌ها

\usepackage{fancyvrb}
\DefineVerbatimEnvironment{terminal}{Verbatim}{baselinestretch=1} %ایجاد متن‌هایی شبیه به خروجی ترمینال

\usepackage{graphicx} %برای درج شکل در متن

\usepackage{xepersian} %افزودن امکان فارسی نویسی و تعریف قلم‌ها
\settextfont[Scale=1.4]{Nazli}
\setlatintextfont[Scale=1.2]{Times New Roman}
\setdigitfont[Scale=1.4]{B Lotus}
\defpersianfont\Yagut[Scale=1.4]{B Yagut}
\defpersianfont\Yagutb[Scale=1.6]{B Yagut}
\defpersianfont\Titr[Scale=7.2]{Titr}
\defpersianfont\Nazanin[Scale=3.6]{B Nazanin}
\defpersianfont\Titrh[Scale=1.8]{Titr}
\defpersianfont\Titrd[Scale=1.2]{Titr}
\defpersianfont\Titrc[Scale=1.4]{Titr}
\deflatinfont\Titles[Scale=1.6]{Times New Roman}
\deflatinfont\Titlec[Scale=1.4]{Times New Roman}

\makeatletter %شروع به تغییرات در قالب پیش‌فرض برای ایجاد قالب دلخواه
\newcommand\Yagutfont[1]{\def\@yagutfont{#1}}
\newcommand\Yagutfontb[1]{\def\@yagutfontb{#1}}
\newcommand\Titrfont[1]{\def\@titrfont{#1}}
\newcommand\Nazaninfont[1]{\def\@nazaninfont{#1}}



\renewcommand\tableofcontents{% تنظیم مکان و فونت کلمه‌ی فهرست مطالب
\newpage
    \if@twocolumn
      \@restonecoltrue\onecolumn
    \else
      \@restonecolfalse
    \fi
    \centerline{\@yagutfontb\bfseries\contentsname
        \@mkboth{%
           \MakeUppercase\contentsname}{\MakeUppercase\contentsname}}%
    \vskip 40.pt
    \@starttoc{toc}%
    \if@restonecol\twocolumn\fi
}

\renewcommand{\@starttoc}[1]{% درج و تنظیم کلمات عنوان و فهرست
 \begingroup
 \hbox to \textwidth{\@yagutfont\bfseries\underline{صفحه}\hfill\underline{عنوان}}
    \@input{\jobname.#1}%
    \if@filesw
      \expandafter\newwrite\csname tf@#1\endcsname
      \immediate\openout \csname tf@#1\endcsname \jobname.#1\relax
    \fi
    \@nobreakfalse
  \endgroup
}

\def\@spart#1{%
    \addcontentsline{toc}{chapter}
       {\@yagutfont#1}
    \thispagestyle{empty}
    {\centering
     \interlinepenalty \@M
     \@titrfont
     \bfseries #1\par}%
    \@endpart
}

\def\@chapter[#1]#2{\ifnum \c@secnumdepth >\m@ne
                         \refstepcounter{chapter}
                         \typeout{\@chapapp\space\thechapter.}
                         \addcontentsline{toc}{chapter}
                                   {\@yagutfont\@chapapp\space\@tartibi\c@chapter: \normalfont#1}
                    \else
                      \addcontentsline{toc}{chapter}{#1}
                    \fi
                    \chaptermark{#1}
                    \addtocontents{lof}{\protect\addvspace{10.pt}}
                    \addtocontents{lot}{\protect\addvspace{10.pt}}
                    \if@twocolumn
                      \@topnewpage[\@makechapterhead{#2}]
                    \else
                      \@makechapterhead{#2}
                      \@afterheading
                    \fi
}

\def\@makechapterhead#1{
  \thispagestyle{empty}
  \if@twocolumn
    \onecolumn
    \@tempswatrue
  \else
    \@tempswafalse
  \fi
  \vspace*{45mm}
  {\centering
     \interlinepenalty \@M
     \@titrfont
     \ifnum \c@secnumdepth >-2\relax
       \bfseries \chaptername\nobreakspace\@tartibi\c@chapter
       \par
       \vskip 60mm
     \fi
     \@nazaninfont
     \bfseries #1\par}
     \newpage
     \vspace*{50.pt}
     \par
}

\def\@makeschapterhead#1{%
  \vspace*{50.pt}%
  {\parindent \z@ \raggedleft
    \normalfont
    \interlinepenalty\@M
    \@yagutfontb \bfseries  #1\par\nobreak
    \vskip 40.pt
  }
}

\renewcommand\section{\@startsection {section}{1}{\z@\\}
                                   {-8.5ex \@plus -1ex \@minus -.2ex}
                                   {2.3ex \@plus.2ex}
                                   {\@yagutfontb\bfseries}
}

\def\thesection{\arabic{section}-\arabic{chapter})}%

\renewcommand\subsection{\@startsection {subsection}{2}{\z@\\}
                                     {-4.25ex\@plus -1ex \@minus -.2ex}
                                     {1.5ex \@plus .2ex}
                                     {\@yagutfont\bfseries}
}

\def\thesubsection{\arabic{subsection}-\arabic{section}-\arabic{chapter})}%  
%تغییرات برای ایجاد subsubsection
\renewcommand{\subsubsection}{\@startsection{subsubsection}{3}{\z@\\}
                                     {-.25ex\@plus -1ex \@minus -.2ex}
                                     {1.5ex \@plus .2ex}
                                     {\@yagutfont\bfseries}}
\renewcommand{\thesubsubsection}{\arabic{subsubsection}. }
\setcounter{subsubsection}{0}%
\setcounter{secnumdepth}{3}
\setcounter{tocdepth}{3}
\def\thesubsubsection{\arabic{subsubsection}-\arabic{subsection}-\arabic{section}-\arabic{chapter})}% پایان تغییرات برای زیر،زیر فصل
\makeatother %پایان تغییرات در قالب پیش‌فرض برای ایجاد قالب دلخواه
\Yagutfont{\Yagut}
\Yagutfontb{\Yagutb}
\Titrfont{\Titr}
\Nazaninfont{\Nazanin}
\SepMark{-} %استفاده از خط تیره به جای نقطه در فاصله‌گذاری شماره‌های بخش، زیربخش و…
